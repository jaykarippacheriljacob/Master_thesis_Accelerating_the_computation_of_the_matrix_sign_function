\section*{Abstract}

The matrix sign function plays a crucial role in computations arising in lattice Quantum Chromodynamics (QCD), particularly for the action $\sgn(Q)x$ of a matrix $Q$ on a vector $x$. Here, $Q$ denotes the symmetrized Wilson-Dirac operator, which is Hermitian when the chemical potential is zero but becomes non-Hermitian otherwise. However it is to be emphasized we are  interested in the non-Hermitian case. Evaluating this function is computationally expensive, and efficient approximation methods are essential. 

A standard approach for approximating the matrix sign function is based on Arnoldi Krylov subspace methods. In this work, we investigate strategies to accelerate the convergence of these methods, focusing on their application to lattice QCD computations. Specifically, we explore combinations of several techniques, including:
\begin{enumerate}
    \item Restarts,
    \item Implicit and explicit deflation,
    \item Polynomial preconditioning, and
    \item Sketching.
\end{enumerate}

Our study aims to provide insights into the interplay of these techniques and their potential to reduce computational costs while maintaining accuracy. Framework for the combination of these algorithms along with numerical experiments and their results are presented to demonstrate the effectiveness of the proposed methods.





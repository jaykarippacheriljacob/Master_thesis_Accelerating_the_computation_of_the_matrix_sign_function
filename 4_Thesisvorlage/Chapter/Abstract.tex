\section*{Abstract}

The matrix sign function arises in computations in lattice QCD. We look at the computation of the action $\text{sign}(Q)x$ of the sign function of the matrix $Q$ on a vector $x$. In our application, $Q$ is the symmetrized Wilson-Dirac operator. This is a Hermitian matrix if the chemical potential is 0; otherwise, it is non-Hermitian. Actually, we will always consider the inverse square root function, since $\text{sign}(Q)x = (Q^2)^{-1/2}Qx$.

The Arnoldi Krylov subspace approximation is the basis method to approximate $\text{sign}(Q)x$. There are several ways to accelerate the convergence of this basic scheme:
\begin{enumerate}
    \item \textbf{Restarts} (in the non-Hermitian case). This avoids having too many inner products in the Arnoldi orthogonalization.
    \item \textbf{Deflation} (explicit and implicit). This makes the matrix better conditioned and thus reduces the number of iterations. Explicit deflation uses the smallest left and right eigenvectors; implicit deflation is present in the thick restart approach of Eiermann and Güttel; see also the \texttt{funm} Matlab code.
    \item \textbf{Polynomial preconditioning.} This also makes the matrix better conditioned and thus reduces the number of iterations. A recent paper on this was published along with the numerical results for QCD on a parallel machine.
    \item \textbf{Sketching.} This is a randomized approach where we save orthogonalizations and sketch the Arnoldi matrix. The relevant paper is by Güttel and Schweitzer.
\end{enumerate}

\textbf{The purpose of the thesis} is to consider the following combination of the above approaches:
\begin{itemize}
    \item $2 + 1$ (as is already done in \texttt{funm})
    \item $2 + 3$ (building on existing work and code of Gustavo)
    \item $2 + 4$ (this is new, but Stefan Güttel just gave a talk on it at a conference in Paris)
\end{itemize}

\textbf{Tasks:}
\begin{enumerate}
    \item Understand and describe the individual methods (1--4).
    \item Describe, formulate algorithmically, and discuss the combined methods ($2+1$, $2+3$, $2+4$).
    \item Test the combined methods, in Matlab on small configurations.
\end{enumerate}





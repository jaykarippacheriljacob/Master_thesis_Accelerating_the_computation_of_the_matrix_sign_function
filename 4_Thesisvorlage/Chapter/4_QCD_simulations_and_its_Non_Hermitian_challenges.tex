\chapter{QCD simulations and its Non-Hermitian challenges}
\label{sec:QCD_sim_and_non_herm_challenges}

% Intro. into QCD
One of the most demanding applications for supercomputers currently is Lattice QCD simulation, where a significant amount of resources are allocated. Quantum chromodynamics (QCD) is a quantum field theory for the strong interaction of the quarks via gluons\cite{30}. This theory is applied to make predictions on masses and resonance spectra on hadrons\cite{29}.

\section{The Wilson-Dirac and the overlap operator in lattice QCD}
\label{sec:wil_dirac_overlap}

% A Brief on the section in QCD where the problem originates.
The governing equation that determines the dynamics of the quarks and the interaction of quarks and gluons is the Dirac equation.

\begin{equation}
    D\psi + m \cdot \psi = \eta
    \label{eq:2.20}
\end{equation}

In the above equation the quark fields are represented by $\psi = \psi(x)$ and $\eta = \eta(x)$, where $x$ denotes the points in space-time,
$x = (x_{0}, x_{1}, x_{2}, x_{3})$\cite{31}. The Dirac operator $D$ in the equation \ref{eq:2.20} represents the gluons and sets the mass of the quarks in the QCD theory. The parameter $m$ is a scalar mass. The Dirac operator can be written as:

\begin{equation}
    D = \sum_{\mu=0}^{3} \gamma_{\mu} \otimes (\partial_\mu + A_\mu)
    \label{eq:2.21}
\end{equation}
where, $\partial_\mu = \partial / \partial x_{\mu}$ and $A$ is the gluon gauge field with the anti-hermitian traceless matrices $A_{\mu}(x)$. The $\gamma$-matrices represent the generators of the Clifford algebra\cite{32}. At a given point $x$, the quark field $\psi$ is expressed by a twelve-component column vector. These column vectors correspond to three colours and four spins, acted upon by $A_{\mu}(x)$ and $\gamma_{\mu}$ respectively.

To align with our study, we rewrite the massless overlap Dirac operator with a non-zero chemical potential ($\mu$) as follows\cite{33}:

\begin{equation}
    D_{ov}(\mu) = 1+\gamma_{5}sgn(H_{w}(\mu))
    \label{eq:2.22}
\end{equation}
where,$H_{w}(\mu) = \gamma_{5}D_{w}(\mu)$, $D_{w}(\mu)$ is the Wilson-Dirac operator at nonzero chemical potential\cite{34, 35} with negative Wilson mass $m_{w} \in (-2, 0)$, $\gamma_{5} = \gamma_{1}\gamma_{2}\gamma_{3}\gamma_{4}$. The Wilson-Dirac operator is a discretization of the Dirac operator on a four-dimensioned lattice given as,

\begin{equation}
    \begin{aligned}
        \relax [D_w (\mu)]_{nm} = \delta_{n,m} \\ 
        &- \kappa \sum_{j=1}^{3} (1 + \gamma_j) U_{n,j} \delta_{n+\hat{j},m}- \kappa \sum_{j=1}^{3} (1 - \gamma_j) U_{n-\hat{j},j}^{\dagger} \delta_{n-\hat{j},m} \\
        &- \kappa (1 + \gamma_4) e^\mu U_{n,4} \delta_{n+\hat{4},m}- \kappa (1 - \gamma_4) e^{-\mu} U_{n-\hat{4},4}^{\dagger} \delta_{n-\hat{4},m}
    \end{aligned}
    \label{eq:2.23}
\end{equation}
where $\kappa = 1/(8+2m_{w})$ and $U_{n,v}$ is the \textit{SU}(3)-matrix associated with the link connecting the lattice site $n$ to $n + \hat{v}$. One of the most important highlights of the Wilson-Dirac operator is that compared to the naive discretization of the derivative operator, it avoids the replication of the fermion species for the continuum Dirac operator.

In the discretized formula \ref{eq:2.23} the non-Hermiticity of the operator arises due to the term $e^{\pm \mu}$. The quark field at each lattice site corresponds to 12 variables: 3 \textit{SU}(3) colour components $\times$ 4 Dirac spinor components. This depicts that the matrix $H_{w}(\mu)$ inside the sign function shifts its properties from Hermitian to non-Hermitian when $\mu \neq 0$. This means we have a new case to be addressed.

The challenge with non-Hermitian matrices lies in the fact that they typically have complex eigenvalues, which complicates the evaluation of the sign function. Therefore, the application of Definition \ref{def:2.8} to Equation \ref{eq:2.22} necessitates the evaluation of the sign of a complex number. Moreover, Definition \ref{def:2.12} offers a clear understanding of the properties that the sign function must satisfy.

We know that for a square matrix $A$, $[\sgn(A)]^2 = I$ needs to concur for a sign function. A short calculation based on the Jordan block canonical form shows that for the above reason the overlap operator $D_{ov}(\mu)$ as defined in Equation \ref{eq:2.22} satisfies the GinspargWilson relation\cite{36}.

\begin{equation}
    {D_{ov},\gamma_{5}}=D_{ov}\gamma_{5}D_{ov}
    \label{eq:2.24}
\end{equation}
For $A$ Hermitian, the polar factor $\text{pol}(A)=A(A^{\dagger}A)^{-1/2}$ of $A$ coincides with $\sgn(A)$.Building upon the above, significant advancements have been made in developing efficient and faster iterative methods for computing the action of the matrix sign function on a vector.  However, for $A$ non-Hermitian, $\sgn(A) \neq pol(A)$ and $pol(A)^{2}\neq I$. Thus, for $\mu \neq 0$, replacing $\sgn(H_w)$ with $\text{pol}(H_w)$ in the definition of the overlap operator in Eq. (1) not only alters the operator but also violates the Ginsparg-Wilson relation, as demonstrated in numerical experiments. We conclude that the definition provided in Equation \ref{eq:2.22} is the correct formulation of the overlap operator for $\mu \neq 0$. This, in turn, generates the motivation for us to explore further iterative methods of sign function of non-Hermitian matrices.
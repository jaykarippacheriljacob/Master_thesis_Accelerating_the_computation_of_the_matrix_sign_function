\chapter{Conclusion}

In this thesis, we have analyzed various combinations of recently developed algorithms, such as quadrature-based sketched Arnoldi, polynomial preconditioning, quadrature-based restarted Arnoldi, and deflation. Based on insights from the reference literature and our numerical experiments, these methods, both individually and in combination, demonstrate significant potential for applications in non-Hermitian matrices.

Our numerical experiments revealed the complexities introduced by non-Hermitian matrices, particularly due to the spread of their spectra on either side of the imaginary axis, as observed in our application matrices. These matrices present unique challenges, including a large number of critical eigenvalues clustering near the imaginary axis in the spectral plots. Furthermore, we noted several important observations, such as the unpredictability and the need for further investigation into convergence rates, as the condition number alone does not reliably predict convergence behaviour for non-Hermitian matrices. Additionally, the unusual behaviour of the 2-norm error plots, caused by the non-orthonormal basis, further complicates the analysis.

From the investigations conducted in this thesis, we conclude that polynomial preconditioning is a powerful tool for finding eigenvalues and eigenvectors. However, the degree of the polynomial must be carefully chosen to balance computational cost and time. Moreover, as established in \cite{4}, deflation serves as an excellent catalyst for accelerating other Krylov subspace methods across all combinations. The primary drawback of deflation lies in the computation of critical eigenvalues and eigenvectors, which becomes increasingly complex as the matrix dimension grows. Consequently, combinations involving LR-deflation are particularly interesting, provided better algorithms for efficiently computing critical eigenvalues and eigenvectors are developed.

Our numerical experiments indicate that the combinations of LR-deflation with polynomial preconditioning and LR-deflation with quadrature-based sketched Arnoldi are the most effective. This is evident from Table ~\ref{tab:inner_pdt_8x4_non_hermitian}, where the number of inner products required for orthogonalization is significantly lower for these combinations to achieve accuracies on the order of $10^{-8}$. Additionally, the LR-deflation combined with quadrature-based sketched Arnoldi demonstrates the least number of matrix-vector multiplications (mvms), the shortest computational time, and steady convergence rates, making it a strong candidate for future research and applications in non-Hermitian problems. Although computationally more expensive, the combination of LR-deflation with polynomial preconditioning is still a compelling choice, as it allows for the reuse of Ritz values and the preconditioning polynomial to condition non-Hermitian matrices for faster evaluation of critical eigenvalues and eigenvectors.

While the other combinations performed as expected, apart from the advantage of restarts for storage efficiency, they do not provide sufficient justification for further exploration. These methods are computationally expensive in terms of inner products for orthogonalization, mvms, and computational time compared to the aforementioned combinations, as observed in our numerical experiments. However, they may still hold promise for specific applications where their unique features align with particular requirements.
%% Version 2023-08-21
%% LaTeX-Vorlage für Abschlussarbeiten
%% Erstellt von Nils Potthoff, ab 2020 erneuert und ausgebaut von Simon Lohmann
%% Lehrstuhl Automatisierungstechnik/Informatik Bergische Universität Wuppertal
%%%%%%%%%%%%%%%%%%%%%%%%%%%%%%%%%%%%%%%%%%%%%%%%%%%%%%%%%%%%%%%%%%%%%%%%%%%%%%%%

%%%%%%%%%%%%%%%%%%%%%%%%%%%%%%%%%%%%%%%%%%%%%%%%%%%%%%%%%%%%%%%%%%%%%%%%%%%%%%%%
%%% DATEI-INFO %%%%%%%%%%%%%%%%%%%%%%%%%%%%%%%%%%%%%%%%%%%%%%%%%%%%%%%%%%%%%%%%%
%%%%%%%%%%%%%%%%%%%%%%%%%%%%%%%%%%%%%%%%%%%%%%%%%%%%%%%%%%%%%%%%%%%%%%%%%%%%%%%%
%%% In dieser Datei werden alle wichtigen Einstellungen der Vorlage gesetzt %%%%
%%% Hier darfst du Werte ändern ;-) %%%%%%%%%%%%%%%%%%%%%%%%%%%%%%%%%%%%%%%%%%%%
%%%%%%%%%%%%%%%%%%%%%%%%%%%%%%%%%%%%%%%%%%%%%%%%%%%%%%%%%%%%%%%%%%%%%%%%%%%%%%%%
%
%%%%%%%%%%%%%%%%%%%%%%%%%%%%%%%%%%%%%%%%%%%%%%%%%%%%%%%%%%%%%%%%%%%%%%%%%%%%%%%%
%%%%%%%%%%%%%%%%%%%%%%%%%%%%%%%%%%%%%%%%%%%%%%%%%%%%%%%%%%%%%%%%%%%%%%%%%%%%%%%%
\ifx\inPreamble\undefined \else %%%% 'MAGIC' %%%%%%%%%%%%%%%%%%%%%%%%%%%%%%%%%%%
%%%%%%%%%%%%%%%%%%%%%%%%%%%%%%%%%%%%%%%%%%%%%%%%%%%%%%%%%%%%%%%%%%%%%%%%%%%%%%%%
%%%%%%%%%%%%%%%%%%%%%%%%%%%%%%%%%%%%%%%%%%%%%%%%%%%%%%%%%%%%%%%%%%%%%%%%%%%%%%%%
%%%%%%% Pflicht-Einstellungen %%%%%%%%%%%%%%%%%%%%%%%%%%%%%%%%%%%%%%%%%%%%%%%%%%

% Bachelor-Thesis, Master-Thesis
\newcommand{\artderarbeit}{Master-Thesis}

% Thema der Thesis (wie in der Aufgabenstellung)
\newcommand{\thema}{This is my thesis}


% Gibt es eine Verlängerung der Bearbeitungszeit?
\setbool{verlaengerung}{false}
% Wenn ja, hier auf "true" setzen und die Datei Verlaengerung.pdf ersetzen

% Eine optionale Danksagung kann in der Datei "Danksagung.tex" formuliert werden
\setbool{danksagung}{false}

% Euer voller Name
\newcommand{\autor}{Max Mustermann}

% Eure Matrikelnummer
\newcommand{\matrikelnummer}{1234567}

% Offizielle Bezeichnung des Studiengangs
\newcommand{\studiengang}{Informationstechnologie}

% Wenn es in Eurem Studiengang keine Schwerpunkte gibt einfach leer lassen
\newcommand{\schwerpunkt}{Systems \& Components}

% Der Lehrstuhl, an dem die Thesis geschrieben wird
\newcommand{\lehrstuhl}{Lehrstuhl für Automatisierungstechnik/Informatik}

% Betreuer? (falls mehrere: mit \\ trennen)
\newcommand{\betreuer}{Vorname Nachname M.Sc.}

% Erstprüfer (siehe Anmeldung)
\newcommand{\prueferA}{Prof. Dr.-Ing. Vorname Nachname}

% Zweitprüfer (siehe Anmeldung)
\newcommand{\prueferB}{Prof. Dr.-Ing. Vorname Nachname}


% Euer Abgabedatum
\newcommand{\abgabedatum}{03. August 1972}

% Ort
\newcommand{\ort}{Wuppertal}

% Schlagwörter, mit denen man das PDF finden kann
\newcommand{\schlagwoerter}{Thesis, Bachelor, Bergische Universität Wuppertal}


%%% Ende der Pflicht-Einstellungen %%%%%%%%%%%%%%%%%%%%%%%%%%%%%%%%%%%%%%%%%%%%%
%%%%%%%%%%%%%%%%%%%%%%%%%%%%%%%%%%%%%%%%%%%%%%%%%%%%%%%%%%%%%%%%%%%%%%%%%%%%%%%%

%%%%%%%%%%%%%%%%%%%%%%%%%%%%%%%%%%%%%%%%%%%%%%%%%%%%%%%%%%%%%%%%%%%%%%%%%%%%%%%%
%%% Layout-Anpassung (Optional) %%%%%%%%%%%%%%%%%%%%%%%%%%%%%%%%%%%%%%%%%%%%%%%%

%%% Zweiseitiges Layout
\setbool{doppelseitig}{true}			% Einseitiger oder doppelseitiger Druck?

%%% Hyperlinks
\setbool{linksMarkieren}{true}			% Anklickbare Links im PDF-Dokument markieren?


%%%%%%%%%%%%%%%%%%%%%%%%%%%%%%%%%%%%%%%%%%%%%%%%%%%%%%%%%%%%%%%%%%%%%%%%%%%%%%%%
%%% Verzeichnisse %%%%%%%%%%%%%%%%%%%%%%%%%%%%%%%%%%%%%%%%%%%%%%%%%%%%%%%%%%%%%%

\setbool{abbildungsverzeichnis}{true} 	% Abbildungsverzeichnis erzeugen?
\setbool{quellcodeverzeichnis}{true}	% Quellcodeverzeichnis erzeugen?
\setbool{tabellenverzeichnis}{true}		% Tabellenverzeichnis erzeugen?
\setbool{symbolverzeichnis}{true}		% Symbolverzeichnis erzeugen?
\setbool{akronymverzeichnis}{true}		% Akronymverzeichnis erzeugen?
\setbool{abkuerzungsverzeichnis}{true}	% Abkürzungsverzeichnis erzeugen?
\setbool{glossar}{true}					% Glossar erzeugen?

\setbool{verzeichnisseImInhaltsverzeichnis}{true} % Verzeichnisse im Inhaltsverzeichnis erwähnen?

\setbool{verzeichnisseZusammenfassen}{true} % Seitenumbruch zwischen den Verzeichnissen deaktivieren?


%%%%%%%%%%%%%%%%%%%%%%%%%%%%%%%%%%%%%%%%%%%%%%%%%%%%%%%%%%%%%%%%%%%%%%%%%%%%%%%%
%%% Kapitelnummerierung %%%%%%%%%%%%%%%%%%%%%%%%%%%%%%%%%%%%%%%%%%%%%%%%%%%%%%%%

\newcommand\kapitelTiefeNummerierung{3} 	  % Wie tief soll nummeriert werden?
\newcommand\kapitelTiefeInhaltsverzeichnis{2} % Bis zu welcher Tiefe sollen die Einträge ins Inhaltsverzeichnis?

% Tiefe Bedeutung
% 0     \chapter
% 1     \section
% 2     \subsection
% 3     \subsubsection
% 4     \paragraph
% 5     \subparagraph


%%% Zusatzerklärung

\setbool{zusatzErklaerung}{false} 		% Hiermit können zusätzliche Erklärungen eingebunden werden (siehe Zusatzerklaerung.tex, wird im Normalfall nicht benötigt)


%%%%%%%%%%%%%%%%%%%%%%%%%%%%%%%%%%%%%%%%%%%%%%%%%%%%%%%%%%%%%%%%%%%%%%%%%%%%%%%%
%%% Farben (Optional) %%%%%%%%%%%%%%%%%%%%%%%%%%%%%%%%%%%%%%%%%%%%%%%%%%%%%%%%%%

\setbool{color}{true} % Farbe für Design-Elemente verwenden (true/false)

\newcommand{\colormodel}{cmyk} % z.B. cmyk, rgb oder gray
% Falls das Tool eurer Druckerei Schwarz-Weiß-Seiten trotz \setbool{color}{false}
% _fälschlicherweise_ als Farbseiten erkennt (das kann ziemlich teuer werden!), 
% könnt ihr hier das Farbmodell der Vorlage umstellen.
%
% So hatten wir schon einmal den Fall, dass eine Thesis-Druckerei die S/W-Seiten
% erst korrekt erkannt hat, wenn das (eigentlich für das Drucken ungeeignete)
% RGB-Modell verwendet wurde...
% 
% Bilder und eingebundene PDFs werden durch diese Einstellung nicht geändert!
%
% Ein paar Beispiele für sinnvolle Werte:
% {cmyk}  - CMYK (Cyan, Magenta, Yellow, Key) ist _das Farbmodell_ für alles was gedruckt wird.
%           Jede professionelle Druckerei kann damit umgehen!
%           => der Standard bei Drucksachen und daher auch in dieser Vorlage
%
% {rgb}   - RGB (Red, Green, Blue) ist ein Farbmodell für Bildschirme etc.
%           Im Gegensatz zum Druck mit Pigmenten (=subtraktive Farbmischung) wird
%           hier mit Licht gearbeitet (Additive Farbmischung). RGB ist daher nicht
%           für den Druck geeignet und wird vor dem Druckvorgang in ein anderes
%           Farbmodell (z.B. CMYK) umgewandelt.
%           => eigentlich falsch, wird in Einzelfällen aber von Druckereien angefordert
%
% {gray}  - Wer möchte, kann auch das Farbmodell "gray" verwenden. Dann werden alle
%           Farben direkt in Graustufen umgerechnet.
%           In diesem Fall wird zusätzlich \setbool{color}{false} empfohlen.
%           So wird die Darstellung von Quelltexten an die nun fehlenden Farben angepasst.
%
% Weblinks zum Thema:
% https://de.wikipedia.org/wiki/CMYK-Farbmodell
% https://de.wikipedia.org/wiki/RGB-Farbraum
% https://www.ctan.org/pkg/xcolor

%%%%%%%%%%%%%%%%%%%%%%%%%%%%%%%%%%%%%%%%%%%%%%%%%%%%%%%%%%%%%%%%%%%%%%%%%%%%%%%%
%%% Seitenränder %%%%%%%%%%%%%%%%%%%%%%%%%%%%%%%%%%%%%%%%%%%%%%%%%%%%%%%%%%%%%%%

\setlength{\bindekorrektur}{1.0cm}
% Wie breit ist der Teil, an dem die einzelnen Blätter 
% mit einander verklebt/anderweitig verbunden werden?

\setlength{\randAussen}{%
	\ifbool{doppelseitig}{%
		2.50cm% 2.5 - 3.4cm (bei doppelseitigem Layout)
	}{%
		1.88cm% 1.88 - 2.5cm (bei einseitigem Layout)
	}%
} 
% Wie viel Rand außen neben dem Text (einseitig: wie viel Rand rechts neben dem Text)
% Innerer bzw. linker Rand wird daraus automatisch berechnet



%%%%%%%%%%%%%%%%%%%%%%%%%%%%%%%%%%%%%%%%%%%%%%%%%%%%%%%%%%%%%%%%%%%%%%%%%%%%%%%%
%%% Literatur-Anpassung (Optional) %%%%%%%%%%%%%%%%%%%%%%%%%%%%%%%%%%%%%%%%%%%%%

% Es gibt drei Arten von Quellen:
%
% Gruppe A : Quellen, die im Text mit \cite{...} zitiert wurden
% Gruppe B : Quellen, die mit \nocite{...} markiert wurden (und sonst nicht zitiert wurden)
% Gruppe C : Quellen, die gar nicht zitiert wurden

\setbool{nichtZitiertInweiterfuehrendeLiteratur}{true}
% nicht zitierte Quellen automatisch mit \nocite{} aufnehmen?
% => d.h. Gruppe C wird automatisch in Gruppe B verschoben 
% => ansonsten wird stattdessen \explizitesNocite aktiv

\setbool{weiterfuehrendeLiteratur}{true}
% Separates Verzeichnis für weiterführende Literatur?
%
% "true": Separates Verzeichnis "Weiterführende Literatur" erstellen
%   Gruppe A => "Literatur"
%   Gruppe B => "Weiterführende Literatur"
% 
% "false": Alles landet in "Literatur"
%   Gruppe A => "Literatur"
%   Gruppe B => "Literatur"

\newcommand{\explizitesNocite}{%
% Wenn bei der Literatur Gruppe B manuell festgelegt werden soll, kann dies hier geschehen:
	\nocite{ARM:AMBA4AXI4StreamProtocol:v1_0}
	\nocite{ARM:AMBA_AXI_and_ACE_Protocol_Specification:E}
	\nocite{AnalogDevices:ADAU1761:rev_C}
	\nocite{Book:PerceptionBasedDataprocessingInAcoustics}
	\nocite{Philips:I2S_BUS_Specification}
	\nocite{Xilinx:PG021:v7_1}
	\nocite{Xilinx:UG473:v1_11}
	\nocite{Xilinx:UG761:v13_1}
	\nocite{Xilinx:UG901:v2016_2}
	\nocite{Xilinx:UG906:v2016_2}
	\nocite{Xilinx:WP231}
	\nocite{Xilinx:XAPP1206:v1_1}
	\nocite{Xilinx:PG109:v9_0}
	\nocite{ARM:NEONProgrammersGuide:v1_0}
	\nocite{Book:TheScientistandEngineersGuidetoDigitalSignalProcessing}
	\nocite{IEEE:754:2008}
	\nocite{SpectrumAndSpectralDensityEstimationByTheDFTincludingAListOfWindowFunctions}
	% ... weitere Quellen ...
}

\setbool{alleAutorenExplizitNennen}{true} % Steuert die Nennung der Autoren im Literaturverzeichnis
% Hier gibt es zwei übliche Varianten:
% true  : Es werden immer alle Autoren genannt.
% false : Bis zu vier Autoren werden mit Namen genannt.
%           Gibt es mehr Autoren, werden die ersten drei genannt und
%           der Rest mit et al. abgekürzt (wie auch im Zitierschlüssel)

%%%%%%%%%%%%%%%%%%%%%%%%%%%%%%%%%%%%%%%%%%%%%%%%%%%%%%%%%%%%%%%%%%%%%%%%%%%%%%%%
%%% Sprach-Anpassung (Optional) %%%%%%%%%%%%%%%%%%%%%%%%%%%%%%%%%%%%%%%%%%%%%%%%
% Eine Liste der mögliche Sprachen finden sich in der Dokumentation zum LaTeX-Paket 'babel'
\newcommand{\hauptsprache}{ngerman} % Sprache, in der das Dokument geschrieben ist. Wird auch für die Verzeichnisnamen etc. verwendet
\newcommand{\weitereSprachen}{ngerman, english, french}% weitere Sprachen, auf die kurzfristig mit \selectlanguage{language} gewechselt wird. (Wenn es mehrere sind werden diese mit Kommata getrennt)

% ngerman - Deutsch nach neuer Rechtschreibung
% english - Englisch
% french - Französisch

%%%%%%%%%%%%%%%%%%%%%%%%%%%%%%%%%%%%%%%%%%%%%%%%%%%%%%%%%%%%%%%%%%%%%%%%%%%%%%%%
\fi %%%%%%%%%%%%%%%%%%%%%%%%%%%%%%%%%%%%%%%%%%%%%%%%%%%%%%%%%%%%%%%%%%%%%%%%%%%%
%%%%%%%%%%%%%%%%%%%%%%%%%%%%%%%%%%%%%%%%%%%%%%%%%%%%%%%%%%%%%%%%%%%%%%%%%%%%%%%%

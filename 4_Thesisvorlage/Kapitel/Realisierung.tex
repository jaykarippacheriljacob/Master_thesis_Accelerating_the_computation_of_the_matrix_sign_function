%% Version 2023-08-21
%% LaTeX-Vorlage für Abschlussarbeiten
%% Erstellt von Nils Potthoff, ab 2020 erneuert und ausgebaut von Simon Lohmann
%% Lehrstuhl Automatisierungstechnik/Informatik Bergische Universität Wuppertal
%%%%%%%%%%%%%%%%%%%%%%%%%%%%%%%%%%%%%%%%%%%%%%%%%%%%%%%%%%%%%%%%%%%%%%%%%%%%%%%%

\chapter{Realisierung}
	\section{title}
		\blindtext
		
		\blindtext
		\blindtext
		
	\section{title}
		\blindtext
		
		\blindtext
		
		\blindtext

		\subsection{title}
			\blindtext
			
			\blindtext
		\subsection{title}
			\blindtext
		
			
			
			\begin{figure}
				\centering
				\begin{tikzpicture}
				\duck[speech={\small Quaak!}, bubblecolor=cyan!20!white, laughing]
				\end{tikzpicture}
				\caption{Bildbeschreibungen sind wichtig, damit der Leser versteht, was er da sieht. Allerdings sollten sie nicht unnötig lang sein -- längere Texte, wie zum Beispiel dieser hier, der ausführlich erläutert, dass auf dem Bild eine gelbe Ente zu sehen ist, welche den Schnabel geöffnet hat und \enquote{Quaak!} sagt, gehören in den normalen Fließtext.}
			\end{figure}
